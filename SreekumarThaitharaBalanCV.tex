% This a XeLaTeX document
\documentclass[]{friggeri-cv}
\usepackage{fontawesome}

\newfontfamily{\FA}{FontAwesome}
\def\twitter{{\FA \faTwitter}}
\def\email{{\FA \faEnvelope}}
\def\linkedin{{\FA \faLinkedinSign}}
\def\homepage{{\FA \faHome}}
\def\github{{\FA \faGithub}}
\def\telephone{{\FA \faPhone}}

\begin{document}
    \header{Sreekumar}{Thaithara Balan}{post-doctoral research associate}

    \begin{aside}% In the aside, each new line forces a line break
        \section{contact}
            13 Ballinger Court,
            Halsey Road,
            Watford, WD18 0JR
            \telephone\ +44 7540757042
            ~
            \href{mailto:tbs1980@gmail.com}{\email\ tbs1980@gmail.com}
            \href{http://tbs1980.github.io}{\homepage\ tbs1980.github.io}
            \href{https://twitter.com/sreekumar_balan}{\twitter\ sreekumar\_balan}
            \href{https://uk.linkedin.com/in/stbalan}{\linkedin\ stbalan}
            \href{https://github.com/tbs1980}{\github\ tbs1980}
        \section{programming}
            C, C++,
            Python, R,
            Fortran, Julia
        \section{parallel / hpc}
            MPI, OpenMP,
            CUDA, OpenCL
        \section{versioning}
            git, mercurial
        \section{build}
            CMake, Waf
        \section{cloud}
            AWS, Rescale
    \end{aside}


    \section{education}
        \begin{entrylist}
            \entry
            {2008 -- 2012}
            {Ph.D. {\normalfont in Physics}}
            {University of Cambridge, UK}
            {Thesis: \emph{Bayesian methods for astrophysical data analysis}
                \begin{itemize}
                    \item Designed a parallel C++ code for performing Hamiltonian Monte Carlo for cosmological data analysis
                    \item Analysed stellar photometry data using a high performance Bayesian object detection algorithm
                    \item Developed a novel method based on wavelets for quantifying non-Gaussanities in cosmological data
                \end{itemize}
            }
            %
            \entry
            {2006 -- 2007}
            {M.Sc. {\normalfont in Physics}}
            {University College London, UK}
            {Thesis: \emph{Orbital parameters of exoplanets from radial velocity measurements}
                \begin{itemize}
                    \item Developed a Bayesian Markov Chain Monte Carlo algorithm for extras-solar planet detection
                \end{itemize}
            }
            %
            \entry
            {1998 -- 2002}
            {B.Tech. {\normalfont in Mechanical Engineering}}
            {National Institute of Technology, Calicut, India}
            {Final year research project:
            \begin{itemize}
                \item Wrote and C code for analysing the stress distribution of an artificial hip-joint using Finite Element Method
            \end{itemize}}
        \end{entrylist}

    \section{experience}
        \begin{entrylist}
            \entry
            {2012 -- Now}
            {University College London}
            {\emph{Post-Doctoral Research Associate}}
            {Achievements:
            \begin{itemize}
                \item Developed a software for performing model selection by optimally propagating uncertainties from cosmological observations
                \item Designed a highly parallel Monte Carlo method for Bayesian computation and benchmarked it on
            various supercomputing platforms.
                \item Offered administrative support for the cosmology group's supercomputer
            \end{itemize}}
            %
            \entry
            {2007 -- 2008}
            {University College London}
            {\emph{Research Assistant}}
            {Achievements:
            \begin{itemize}
                \item Created a C++ library for stochastically simulating a billion galaxy images
                \item Designed an Object Oriented Markov Chain Monte Carlo software for Bayesian exoplanet detection
            \end{itemize}}
        \end{entrylist}

    \section{awards}
        \begin{entrylist}
            \entry
            {2009 -- 2011}
            {Isaac Newton Studentship}
            {University of Cambridge}
            {Awarded to two out of more than hundred applicants}
            %
            \entry
            {2007}
            {Royal Astronomical Society Grant (£2000)}
            {University College London}
            {For the development of a statistical software package for analysing radial velocity data of stars}
        \end{entrylist}

    \section{publications}
        \begin{entrylist}
            \entry
            {h-index:11}
            {Three first-author and more than five second-author publications}
            {}
            {}
        \end{entrylist}
\end{document}
